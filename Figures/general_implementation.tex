% dvisvgm --pdf name
% latexmk -C
\documentclass[preview]{standalone}
\usepackage{tikz}
\usepackage{amsmath}
\usepackage{svg}
\usepackage{xcolor}
\usepackage[cache=false]{minted}

\usetikzlibrary{shapes.geometric, shapes.arrows}
\usetikzlibrary {patterns,patterns.meta} 
\usetikzlibrary{quantikz2}
     \tikzset{operator/.append style={fill=white!0}}
\begin{document}




    \begin{tikzpicture}[transform shape, every node/.append style={}, node distance = 2cm]
        \node (board) [scale=0.46] at (0,0) {\includesvg{position1.svg}};
        \node[above] at (board.north) {\Huge Board};

        \node (quantcirc) at (9, 3) {
            \begin{quantikz}[scale=1.5]
                \lstick{$\ket{0}_0$} & \setwiretype{q}& & \\
                \lstick{$\ket{0}_1$} &\setwiretype{q} &&\\
                \lstick{$\ket{0}_2$} &\setwiretype{q} &\gate{X} & \rstick{\ket{1}}\\[-1em]
                &\setwiretype{n}  &  \vdots& & &\\[-0.5em]
                \lstick{$\ket{0}_{31}$} &\setwiretype{q}  &&
            \end{quantikz}};

        \node[above] at (quantcirc.north) {\Huge Quantum circuit};


        \node (python) [below] at (quantcirc.south) {\Huge Python};
        \node (boardtext) [below] at (python.south) {\Large Board};
        \node (boardcode) [below] at (boardtext.south) {
            \begin{tabular}{l}
                board = [\textcolor{blue}{None}]*32 \\
                board[2] = \{\\
                    \quad \textcolor{orange}{'color'} : \textcolor{blue}{red}, \\
                    \quad \textcolor{orange}{'probability'} : \textcolor{blue}{1}, \\
                    \quad \textcolor{orange}{'pawn'} : \textcolor{blue}{1} \\
                    \}
            \end{tabular}
        };
        \node (quantumtext) [below] at (boardcode.south) {\Large Quantum};
        \node (quantumcode) [below] at (quantumtext.south) {
            \begin{tabular}{l}
                q = QuantumRegister(32, \textcolor{orange}{'q'}) \\
                circuit = QuantumCircuit(q)\\
                \\
                circuit.x(q[2])
            \end{tabular}
        };

        % \draw[] (3.5, -7) -- +(1,0) -- +(1, -2) -- +(1.4, -2) -- +(0.5, -2.5) -- +(-0.4, -2) -- +(0, -2) -- +(0,0);
        \node(arrow) [single arrow, draw, shape border uses incircle, minimum height = 2 cm, line width = 0.1 cm] at (4, -7.5) [shape border rotate = -90]{};

        \node(left_dice) [rectangle, draw, minimum width = 1.5 cm, minimum height = 1.5 cm, rounded corners, line width = 0.15 cm] at ($(arrow.west) +(-2cm,0)$) {};
        \node(right_dice) [rectangle, draw, minimum width = 1.5 cm, minimum height = 1.5 cm, rounded corners, line width = 0.15 cm] at ($(arrow.east) +(2cm,0)$) {};
        \node[circle, fill] at (right_dice) {};
        \node[circle, fill] at (left_dice) {};
        \node[circle, fill] at ($(left_dice) + (0.35, 0.35)$) {};
        \node[circle, fill] at ($(left_dice) + (0.35, -0.35)$) {};
        \node[circle, fill] at ($(left_dice) + (-0.35, 0.35)$) {};
        \node[circle, fill] at ($(left_dice) + (-0.35, -0.35)$) {};

        \node (board2) [scale=0.46] at (0,-15) {\includesvg{position2.svg}};
        \node[above] at (board2.north) {\Huge Board};

        \node (quantcirc2) at (9.5, -12) {
            \begin{quantikz}[scale=1]
                  \lstick{$\ket{0}_2$} &\gate{X}  &\ctrl{1}  & \swap{3} &         & \rstick{\ket{0}}\\
                   \lstick{$\ket{0}_3$} & &\gate{H}  &          &\ctrl{2} & \rstick[wires=3]{$\ket{01}+\ket{10}$} \\[-1em]
                   \setwiretype{n} &\vdots             &                &          &          &         & \\[-0.5em] 
                  \lstick{$\ket{0}_7$} &  &          & \swap{-3} &\targ{ } & 
            \end{quantikz}};

        \node[above] at (quantcirc2.north) {\Huge Quantum circuit};

        \node (python2) [below] at (quantcirc2.south) {\Huge Python};
        \node (boardtext2) [below] at (python2.south) {\Large Board};
        \node (boardcode2) [below] at (boardtext2.south) {
            \begin{tabular}{l}
                board[2] = \textcolor{blue}{None} \\ 
                board[3, 7] = \{\\
                    \quad \textcolor{orange}{'color'} : \textcolor{blue}{red}, \\
                    \quad \textcolor{orange}{'probability'} : \textcolor{blue}{0.5}, \\
                    \quad \textcolor{orange}{'pawn'} : \textcolor{blue}{1}\} \\
                
            \end{tabular}
        };
        \node (quantumtext2) [below] at (boardcode2.south) {\Large Quantum};
        \node (quantumcode2) [below] at (quantumtext2.south) {
            \begin{tabular}{l}
                circuit.x(q[2], q[3]) \\
                circuit.ch(q[2], q[3]) \\
                circuit.swap(q[2], q[7]) \\
                circuit.cx(q[2], q[7]) \\
            \end{tabular}
        };
        \end{tikzpicture}
    \end{document}